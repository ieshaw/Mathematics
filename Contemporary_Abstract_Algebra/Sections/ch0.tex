5. Show that if $a$ and $b$ are positive integers, then $ab = \text{lcm}(a,b) \bullet \text{gcd}(a,b)$.

\begin{proof}
By the Fundamental Theorem of Arithmetic (Theorem 0.3 in the book) that $a$ is the product of primes, and by definition the greatest common divisor is a product of a subset of these primes
\[ a = \text{gcd}(a,b)\prod_{i=1}^{n} p_{a,i}\]
Since by this definition
\[\{\emptyset\} = \{p_{a,i}\} \cap \{p_{b,i}\}  \]
Thus 
\[\text{lcm}(a,b)= \text{gcd}(a,b)\prod_{i=1}^{n} p_{a,i} \prod_{j=1}^{m} p_{b,j}\]
Therefore
\begin{align*}
\text{lcm}(a,b) \text{gcd}(a,b) &= \text{gcd}(a,b)\text{gcd}(a,b)\prod_{i=1}^{n} p_{a,i} \prod_{j=1}^{m} p_{b,j} \\
&= \big(\text{gcd}(a,b) \prod_{i=1}^{n} p_{a,i} \big)\big( \text{gcd}(a,b)\prod_{j=1}^{m} p_{b,j}\big) \\
&= ab
\end{align*}
\end{proof}

7. If $a$ and $b$ are integers and $n$ is a positive integer, prove that $ a \;\mathrm{mod}\; n = b \;\mathrm{mod}\; n$ if and only if $n$ divides $a-b$.

\begin{proof}
Forward: If $ a \;\mathrm{mod}\; n = b \;\mathrm{mod}\; n$, then $n$ divides $a-b$. 

So we know that
\begin{align*}
m_a, m_b, r, &\in \mathbb{Z} \\
a &= m_an + r \\
b &= m_bn + r \\
\end{align*}
Doing some artihmetic
\[ a - b = (m_a - m_b)n \implies n | (a - b)\]

Backward: If $n$ divides $a-b$, then $ a \;\mathrm{mod}\; n = b \;\mathrm{mod}\; n$.
\begin{align*}
mn &= a - b \\
a &= mn +b 
\end{align*}
We also know by the Division Algortihm
\begin{align*}
b &= m_bn + r \\
r &= b\;\mathrm{mod}\; n
\end{align*}
Thus with substitution
\begin{align*}
a &= mn + m_bn + r \\
&= n*(m - m_b) +r \implies r = a\;\mathrm{mod}\; n
\end{align*}
\end{proof}

11. Let $n$ and $a$ be positive integers and let $d=\text{gcd}(a,b)$. Show that the equation $ax\mod n = 1$ has a solution if and only if $d=1$.

\begin{proof}
Forward: If the equation $ax\mod n = 1$ has a solution, then $d=1$.

By Theorem 0.2 (p.4),
 \[ d=\text{gcd}(a,b) \implies \exists s,t \in \mathbb{Z} \text{ s.t. } as + nt = d\].

 Also, by definition

 \begin{align*}
 ax\mod n = 1 \implies \exists m_n \in \mathbb{Z} \text{ s.t. } ax &= nm_n + 1 \\
 ax - nm_n &= 1
 \end{align*}

Further reading of Theorem 0.2 says that $\text{gcd}(a,b)$ is the smallest possible integer of the form $as + nt = d$. Since 1 is the smallest possible integer, 
\[1 = \text{gcd}(a,b)\]

Backward: If $d=1$, then the equation $ax\mod n = 1$ has a solution.

This is pretty straight forward
\begin{align*}
\text{gcd}(a,b) = 1 \implies \exists s,t \in \mathbb{Z} \text{ s.t. } as + nt &= 1 \\
as &= n (-t) + 1\\
as \mod n &= 1
\end{align*}
\end{proof}
